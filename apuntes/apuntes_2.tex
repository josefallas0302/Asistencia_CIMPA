\documentclass{article}
\usepackage[utf8]{inputenc}
\usepackage{amssymb, amsmath, amsbsy} % simbolitos
%\usepackage{amssymb}
\usepackage{upgreek} % para poner letras griegas sin cursiva
\usepackage{cancel} % para tachar
\usepackage{mathdots} % para el comando \iddots
\usepackage{mathrsfs} % para formato de letra
\usepackage{stackrel} % para el comando \stackbin

\begin{document}

\section{E.D Lineal}

\subsection{Definicion}
	E.D lineal en y ($frac{dy}{dx}$),   $a(x)y'+ b(x)y = c(x)$. \\
	E.D lineal en x ($frac{dx}{dy}$ cambio de x con respecto a y),	$a(y)x'+ b(y)x'= c(y)$. \\

Para resolver E.D lineal necesitamos encontrar un factor integrante	

\subsection{Teorema (Cálculo de factor integrante)}

		La E.D lineal y'+p(x)y  = q(x) tiene al menos un factor integrante de forma $u = e^{\int p(x)dx}$

\subsection{Demostración}

	\begin{equation}
		uy'+ up(x)y = \underbrace{u}_{factor integrante}q(x)
		\label{eq:factor_integrante}
	\end{equation}

Queremos escribirlo de la siguiente forma $(u \cdot y )' = u \cdot g()x \ = \ uy' + yu' = uq(x) $, si comparamos la ecuación anterior con \ref{eq:factor_integrante}, solo necesitamos que $u' = up(x)$ . 

Suponemos que $u = u(x) \neq =0$ y tenemos lo siguiente:

\begin{eqnarray}
\nonumber u' = up(x) \\
\nonumber \frac{du}{dx} = up(x) \\
\nonumber \frac{du}{u} = p(x) \\
\nonumber ln|u| = \int  p(x) dx + C
\end{eqnarray}

Como solo se requiere un factor integrante, entonces c = 0.

\begin{eqnarray}
\nonumber e^{ln|u|} = \mp e^{\int p(x) dx}\\
\nonumber u =  e^{\int p(x)dx}\\
\nonumber \underbrace{u = e^{\int p(x) d(x)}_{Factor integrante} }
\end{eqnarray}


\underline{Como lo usamos para resolver E.D ?} \\
\underline{E.D Lineal:} 
\begin{equation}
a(x)y' + b(x)y = c
\end{equation}

Donde a(x) , b(x) y c(x) , son coeficientes que pueden ser constantes a funciones de x .\\

\underline{Normalizamos}

\begin{eqnarray}
\nonumber y' +\frac{b(x)}{a(x)} = \frac{c(x)}{a(x)}\\
\nonumber P(x) = \frac{b(x)}{a(x)}, \  Q(x) = \frac{c(x)}{a(x)} \\
\nonumber y' + p(x)y =q(x)
\end{eqnarray}

Calculamos el factor integrante que sabemos que es $e^{\int P(x) dx}$y multiplicamos la E.D por el factor integrante.


\begin{eqnarray}
\nonumber \underbrace{e^{\int p(x) dx}}_{u(x)}y' + \underbrace{e^{\int p(x) dx}}_{u(x)} p(x)y = \underbrace{e^{\int P(x) dx}}_{u(x)}q(x) \\
\nonumber \frac{d}{dx}\left(e^{\int P(x) dx} \cdot y \right) = e^{\int P(x) dx} \cdot q(x)\\
\nonumber d\left(e^{\int P(x) dx} \cdot y \right) = e^{\int P(x) dx} \cdot q(x) dx\\
\nonumber \int d \left( e^{\int P(x) dx} \cdot y \right) = \int e^{\int P(x) dx} q(x) \\
\nonumber e^{\int P(x) dx} y = \int e^{\int P(x) dx} q(x)) dx \\
\nonumber \underbrace{y = e^{- \int P(x) dx} \int e^{\int P(x) dx}  q(x) dx}_{Formula de solucion general}
\end{eqnarray}


\subsection{Ejemplo}

\begin{equation}
y'- 2xy = x
\end{equation}

\begin{eqnarray}
\nonumber a(x) = 1, \ b(x) &=& -2x, \ c(x)=x , \rbrace coeficientes \\
\nonumber p(x) = -2x, \  q(x) &=& x \\
\nonumber u = e^{\int p(x) dx} &=& e^{\int -2x dx} = e^{-x^2} \rbrace Factor integrante\\
\nonumber e^{-x^2}y' -e^{-x^2}(2xy) &=& xe^{-x^2} \\
\nonumber \left( e^{-x^2} \cdot y \right) ' &=& e^{-x^2} \cdot x \\
\nonumber e^{-x^2} y' + y (-2x)  e^{-x^2} &=& x e^{-x^2}\\
\nonumber \int \left( e^{-x^2} \cdot y \right)' &=& \int x e^{-x^2} dx \\
\nonumber \left( e^{-x^2} \cdot y \right)'  &=& \int x e^{-x^2} dx  \\
\nonumber & & u = -x^2 \\
\nonumber & & du	= -2x dx \\
\nonumber & & -\frac{du}{2} = x dx \\
\nonumber e^{-x^2 } y &=& \int \frac{1}{2} e^{u} du\\
\nonumber e^{-x^2} y &=& \frac{1}{2} e^{-x^u} + C \\
\nonumber y &=& e^{x^2} \left(  \frac{1}{2} e^{-x^u} + C \right) \\
\nonumber y &=& \frac{-1}{2} +  C e^{x^2} \leftarrow Solucion \  General. \\
\end{eqnarray}

\subsection{Ejemplo}

 \begin{equation}
 xy'' + 2y' = e^x , x >  0 
 \end{equation}

\begin{eqnarray}
\nonumber  E \left( x,y',y'' \right) \leftarrow y \ ausente  \\
\nonumber  v = y' ,  v'= y''  \\
\nonumber xv' + 2v = e^x  \leftarrow E.D \ Lineal \\
\nonumber  a(x), \ b(x) = 2, \ c(x) = e^x \\
\nonumber \underline{Normalizamos} \\
\nonumber v' +  \frac{2}{x} v = \frac{e^x}{x}  , \ p(x)  = \frac{b(x)}{a(x)} = \frac{2}{x} , \ q(x) = \frac{c(x)}{a(x)} = \frac{e^x}{x} \\
\nonumber u = e{\int P(x) dx} =  e^{2 \int \frac{1}{x} dx} = e^{2 ln |x|} = x^2 \\
\nonumber x^2 v' +  \frac{2}{x} \cdot x^2v = \frac{e^x}{x} \cdot x^2 \\
\nonumber x^2 v' + 2xv = xe^x \\
\nonumber  \left( x^2 v \right)' = xe^x \\
\nonumber \int \left( x^2 v \right)' = \int xe^x \\
\nonumber x^2 v = \underbrace{\int xe^x}_{por partes} \\
\nonumber & & u = x ,  \ du   dx \\
\nonumber & & dv = e^x dx, \ v = e^x \\
\nonumber & & xe^x - \int e^x dx \\
\nonumber & & = xe^x - e^x \\
\nonumber & & = e^x \left( x-1 \right) + c \\
\nonumber v = \frac{e^x}{x^2} \left( x-1 \right) + cx^{-2} \\
\nonumber y'= \frac{e^x}{x^2} \left( x-1 \right) + cx^{-2}\\
\nonumber \int dy = \int \left( \frac{e^x}{x} -\frac{e^x}{x^2} + \frac{c}{x^2} \right) dx \\ 
\nonumber y = \int  \frac{e^x}{x} dx - \underbrace{\int \frac{e^x}{x^2} dx}_{Por \ partes} + c \int \frac{1}{x^2} dx \\
\nonumber Por \ partes \\
\nonumber u= e^x , \ du = e^xdx \\
\nonumber dv = \frac{-1}{x^2} dx v = \frac{1}{x} \\
\nonumber - \int \frac{e^x}{x^2} dx = \frac{e^x}{x} - \int \frac{e^x}{x}dx \\
\nonumber y = \int \frac{e^x}{x}dx +\frac{e^x}{x} - \int \frac{e^x}{x} dx - \frac{c}{x} + D
\nonumber y = \frac{e^x}{x} - \frac{c}{x} +D\\
\nonumber y'(x) = x^{-1} \cdot e^x - cx^{-1} + D \\
\nonumber = x^{-1} e^x + e^x \left( -x^{-2} \right) + cx^-2\\
\nonumber y'(x) = \frac{e^x}{x} -\frac{e^x}{x^2} + \frac{c}{x^2}  \\ 
\end{eqnarray}



\end{document}



